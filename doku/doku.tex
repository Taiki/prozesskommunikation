\documentclass[12pt, a4paper]{scrartcl}
\usepackage[utf8]{inputenc}

\usepackage{graphicx}
\usepackage{fancyhdr}
\usepackage{url}
\usepackage{listings}
\usepackage{booktabs}
\usepackage{palatino,avant}
\usepackage{amsmath,amssymb,amsthm,mathabx,mathrsfs,stmaryrd}
%\usepackage{txfonts}
\usepackage{pb-diagram}
\usepackage[pdftex,bookmarks=true,bookmarksnumbered=true,bookmarksopen=true,colorlinks=true,filecolor=black,
                linkcolor=red,urlcolor=blue,plainpages=false,pdfpagelabels,citecolor=black,
                pdftitle={BTS-Dokumentation},pdfauthor={Michael Stapelberg}]{hyperref}
\begin{document}
\pagestyle{fancy}
\chead{BTS}
% New paragraph
\newcommand{\np}{\bigskip\noindent}

% No indention at new paragraphs
\setlength{\parindent}{0pt}

\section{Grundgerüst}

Das Grundgerüst wird in Teilaufgabe 2, 3 und 4 verwendet und startet die
verschiedenen Prozesse. Dies wird durch die Funktion \texttt{fork\_child()}
erledigt, welche als Parameter zwei Zeiger auf Funktionen erwartet. Der erste
Zeiger zeigt auf eine Funktion, welche die eigentlich Arbeit des Prozesses
erledigt, der zweite Zeiger auf eine Aufräumfunktion. Das Definieren und
Benutzen der \texttt{fork\_child}-Funktion erspart uns viel redundanten Code.
\np

Die Funktion ruft zunächst \texttt{fork} auf, um den Programmverlauf zu
teilen. Im Kindprozess wird dann ein Signalhandler für das Signal
\texttt{SIGINT} (Unterbrechung durch Ctrl-C) eingerichtet (hierfür wird der
Aufräum-Funktionszeiger genutzt). Anschließend wird der erste Funktionszeiger
aufgerufen. Dieser Aufruf kehrt nicht mehr zurück, da in allen Prozessen eine
Endlosschleife enthalten ist.
\np

Im Elternprozess hingegen wird nach erfolgreichem Starten aller 4 Prozesse ein
Signalhandler für \texttt{SIGTERM} (Beenden des Prozesses, z.B. durch
\texttt{kill}) registriert, welcher seinerseits an die 4 Prozesse ein
\texttt{SIGINT} sendet. Dadurch werden alle Kindprozesse auch beim Beenden des
Elternprozesses korrekt beendet und geben ihre Resourcen frei.

\subsection{Conv-Prozess}

Der \texttt{Conv}-Prozess generiert mithilfe von \texttt{rand} eine
ganzzahlige Zufallszahl. Dies entspricht dem Einlesen eines Werts von einem
Analog-Digital-Wandler. Dieser Wert wird anschließend zum \texttt{Log}- und zum
\texttt{Statistik}-Prozess gesendet.
\np

Der \texttt{Conv}-Prozess hat per se keinerlei dynamische Resourcen, die
freigegeben werden müssten.

\subsection{Log-Prozess}

Der \texttt{Log}-Prozess öffnet zunächst die Datei \texttt{log.txt} zum
Schreiben (er überschreibt sie, sofern sie bereits vorhanden ist) und empfängt
dann Nachrichten. Die empfangenen Daten werden -- ein Datenwert pro Zeile --
direkt in die Datei geschrieben. Es wird hierbei nicht via \texttt{fflush}
sichergestellt, dass die Ausgabe tatsächlich auf die Festplatte geschrieben
wurde. Stattdessen wird darauf vertraut, dass \texttt{fclose} beim Beenden des
Programms aufgerufen wird, wobei alle Daten aus dem Puffer auf die Platte
geschrieen werden.
\np

Der \texttt{Log}-Prozess gibt via \texttt{fclose} das Dateihandle frei, welches
er beim Start geöffnet hat.

\subsection{Statistik-Prozess}

Der \texttt{Statistik}-Prozess empfängt je 5 Datenwerte und bildet anschließend
das arithmetische Mittel über die empfangenen Werte. Das gebildete Mittel
schickt er wiederum an den \texttt{Monitor}-Prozess.
\np

Da dieser Prozess ausschließlich mit statischen Puffern arbeitet hat er per se
ebenfalls keine freizugebenden Resourcen bei Prozessende.

\subsection{Monitor-Prozess}

Der \texttt{Monitor}-Prozess empfängt die Mittelwerte und gibt sie direkt auf
dem Bildschirm aus. Dass die Ausgabe tatsächlich sofort erfolgt wird via
\texttt{fflush} sichergestellt.
\np

Dieser Prozess hat ebenfalls per se keine freizugebenden Resourcen.

\clearpage
\section{Aufgabe 4}

Bei Aufgabe 4 ging es um die Implementation der Prozesskommunikation über
Shared Memory und Semaphoren. Hierbei haben wir uns für eine einfache
Warteschlangen-Implementation (Queue) entschieden. Diese benutzt einen
Ringpuffer mit $255$ Elementen, wobei immer nur das erste Element verarbeitet
wird. Zum Schreiben in den Ringpuffer muss der Semaphor gesperrt werden, was zu
jedem Zeitpunkt nur einem Prozess gelingt.

\subsection{Datenstruktur}

Am Anfang des Speichersegments steht ein Header, in welchem der Index des
aktuellen Elements im Ringpuffer gespeichert wird. Weiterhin befindet sich hier
der Semaphor zum Sperren der Schreibzugriffe (ein Schreibzugriff ist auch das
Verarbeiten einer Nachricht, sodass der Ringpuffer-Index auf das nächste
Element verschoben wird).
\np

Die Einträge im Ringpuffer wiederum enthalten eine Richtung (\texttt{dir} für
Direction), welche angibt, von welchem Prozess die Nachricht stammt und an
welchen sie geht. Weiterhin wird der Datenwert selbst als \texttt{uint8\_t},
also als 1-Byte-Integer ohne Vorzeichen gespeichert.

\subsection{Initialisierung: queue\_init}

\texttt{queue\_init} erstellt via \texttt{shm\_open} das Shared Memory-Segment, setzt die Größe und bildet es in den Arbeitsspeicher des Programms ab.

\end{document}
